\documentclass[]{article}
\usepackage{amsmath,amssymb,amsthm}
\usepackage[utf8]{inputenc}
\usepackage{lmodern}
%\usepackage{circuitikz}
\makeatletter
\@ifpackageloaded{tex4ht}{
    \def\pgfsysdriver{pgfsys-tex4ht.def}
}
\makeatother
\usepackage{pgfplots}
\usepackage{pgfplotstable}
\usepackage{pgf,tikz}
\usetikzlibrary{shapes,backgrounds,positioning,matrix,decorations}

\usepackage{siunitx}
\usepackage{python}
\usepackage{ifxetex,ifluatex}
\usepackage{listings}
% \usepackage[xindy,acronym,nomain,toc]{glossaries}
% \makeglossaries
%\usepackage[xindy]{imakeidx}
%\makeindex
\setlength{\parskip}{3mm}
\newtheorem{axiom}{Axiom}
\newtheorem{definition}{Definition}
\newtheorem{comment}{Comment}
\newtheorem{example}{Example}
\newtheorem{lemma}{Lemma}
\newtheorem{property}{Property}
\newtheorem{problem}{Problem}
\newtheorem{remark}{Remark}
\newtheorem{theorem}{Theorem}
\newtheorem{script}{Script}

\usepackage{fixltx2e} % provides \textsubscript
% use upquote if available, for straight quotes in verbatim environments
\IfFileExists{upquote.sty}{\usepackage{upquote}}{}
\ifnum 0\ifxetex 1\fi\ifluatex 1\fi=0 % if pdftex
  \usepackage[utf8]{inputenc}
\else % if luatex or xelatex
  \ifxetex
    \usepackage{mathspec}
    \usepackage{xltxtra,xunicode}
  \else
    \usepackage{fontspec}
  \fi
  \defaultfontfeatures{Mapping=tex-text,Scale=MatchLowercase}
  \newcommand{\euro}{€}
\fi
% use microtype if available
\IfFileExists{microtype.sty}{\usepackage{microtype}}{}
\ifxetex
  \usepackage[setpagesize=false, % page size defined by xetex
              unicode=false, % unicode breaks when used with xetex
              xetex]{hyperref}
\else
  \usepackage[unicode=true]{hyperref}
\fi
\hypersetup{breaklinks=true,
            bookmarks=true,
            pdfauthor={Dilawar Singh},
            pdftitle={Lecture 6, Question 1},
            colorlinks=true,
            citecolor=blue,
            urlcolor=blue,
            linkcolor=magenta,
            pdfborder={0 0 0}}
\urlstyle{same}  % don't use monospace font for urls
\setlength{\parindent}{0pt}
\setlength{\parskip}{6pt plus 2pt minus 1pt}
\setlength{\emergencystretch}{3em}  % prevent overfull lines
\setcounter{secnumdepth}{5}

\title{Lecture 6, Question 1}
\author{Dilawar Singh}
\date{\today}
\usepackage[margin=15mm]{geometry}
\usepackage{graphviz}

\begin{document}
\maketitle

\begin{quote}
Design a simple neural network that would tell you if you would prefer
tea, coffee, or milkshake. Think about internal inputs to the network
(hunger, sleepiness), about external inputs (cost, attractiveness, time)
and anything else. Suggest weights that could succeed at this.
\end{quote}

All inputs are discretized and normalized {[}0, 1.0{]}. For
\emph{day\_time} 0 means morning and 1.0 means sleeping time, hunger
level 0 is not hungry and 1 means maximum possible hunger, and same for
sleepiness. Typical weights betweem 0 and 100 are put on edges.

External inputs cost, time, and energy are vector of length equal to the
number of input. They put equal weights on all output. No hidden layer
is used in this network.

At each output, value is linear sum of weights multiplied by input
value. In the end, output with highest value is chosen e.g.~for input
value leepiness = 0.8, hunger = 0.3, cost = \{0.1, 0.11, 0.3\}, time=
\{0.1,0.2,0.3\}, energy = \{0.1, 0.1, 0.3\}, we have milkshake = 22,
coffee = 85, and tea = 46 suggesting to have coffee.

\digraph[scale=0.5]{hnn}{

    hunger -> milkshake [label = "90"];
    hunger -> {tea, coffee} [label="10"];

    sleepiness -> tea [label=60];
    sleepiness -> coffee [label=100];
    sleepiness -> milkshake [label="-50"];

    cost -> {coffee, milkshake, tea } [label=-10];
    time -> {coffee, milkshake, tea} [label=-30];
    energy -> {coffee, milkshake, tea} [label=50];
}

\end{document}
