\documentclass[]{article}
\usepackage{amsmath,amssymb,amsthm}
\usepackage[utf8]{inputenc}
\usepackage{lmodern}
%\usepackage{circuitikz}
\makeatletter
\@ifpackageloaded{tex4ht}{
    \def\pgfsysdriver{pgfsys-tex4ht.def}
}
\makeatother
\usepackage{pgfplots}
\usepackage{pgfplotstable}
\usepackage{pgf,tikz}
\usetikzlibrary{shapes,backgrounds,positioning,matrix,decorations}

\usepackage{siunitx}
\usepackage{python}
\usepackage{ifxetex,ifluatex}
\usepackage{listings}
% \usepackage[xindy,acronym,nomain,toc]{glossaries}
% \makeglossaries
%\usepackage[xindy]{imakeidx}
%\makeindex
\setlength{\parskip}{3mm}
\newtheorem{axiom}{Axiom}
\newtheorem{definition}{Definition}
\newtheorem{comment}{Comment}
\newtheorem{example}{Example}
\newtheorem{lemma}{Lemma}
\newtheorem{property}{Property}
\newtheorem{problem}{Problem}
\newtheorem{remark}{Remark}
\newtheorem{theorem}{Theorem}
\newtheorem{script}{Script}

\usepackage{fixltx2e} % provides \textsubscript
% use upquote if available, for straight quotes in verbatim environments
\IfFileExists{upquote.sty}{\usepackage{upquote}}{}
\ifnum 0\ifxetex 1\fi\ifluatex 1\fi=0 % if pdftex
  \usepackage[utf8]{inputenc}
\else % if luatex or xelatex
  \ifxetex
    \usepackage{mathspec}
    \usepackage{xltxtra,xunicode}
  \else
    \usepackage{fontspec}
  \fi
  \defaultfontfeatures{Mapping=tex-text,Scale=MatchLowercase}
  \newcommand{\euro}{€}
\fi
% use microtype if available
\IfFileExists{microtype.sty}{\usepackage{microtype}}{}
\ifxetex
  \usepackage[setpagesize=false, % page size defined by xetex
              unicode=false, % unicode breaks when used with xetex
              xetex]{hyperref}
\else
  \usepackage[unicode=true]{hyperref}
\fi
\hypersetup{breaklinks=true,
            bookmarks=true,
            pdfauthor={Dilawar Singh},
            pdftitle={Lecture 3, Question 1},
            colorlinks=true,
            citecolor=blue,
            urlcolor=blue,
            linkcolor=magenta,
            pdfborder={0 0 0}}
\urlstyle{same}  % don't use monospace font for urls
\setlength{\parindent}{0pt}
\setlength{\parskip}{6pt plus 2pt minus 1pt}
\setlength{\emergencystretch}{3em}  % prevent overfull lines
\setcounter{secnumdepth}{5}

\title{Lecture 3, Question 1}
\author{Dilawar Singh}
\date{\today}
\usepackage[margin=15mm]{geometry}
\usepackage{pgfplots}
\usepackage{circuitikz}

\begin{document}
\maketitle

\begin{quote}
Draw the following current-voltage curves:
\end{quote}

\begin{quote}
\begin{enumerate}
\def\labelenumi{\alph{enumi}.}
\itemsep1pt\parskip0pt\parsep0pt
\item
  A resistor
\item
  A resistor in series with a battery of -80 mV
\item
  A resistor in series with a battery of -80 \textgreater{} mV, where
  the resistance is infinity below -60 mV, is 2R from -60 to 0 mV, and
  is R above 0 mV.\\
\item
  A potassium channel, assuming no inactivation. Assume that the K
  channel is closed below -60 mV and has a reversal potential of -80 mV
  and has an open conductance of R, for all potentials above 0 mV.
\end{enumerate}
\end{quote}

\section{A resistor}\label{a-resistor}

\begin{figure}[!ht]
\begin{center}
\begin{tikzpicture}[scale=1]
    \begin{axis}[
        title={$V_R(t) = R * i_R(t)$}
        , xlabel=$V_R(t)$
        , ylabel=$i_R(t)$
        , legend pos=north west
        , domain=-1:1
        , samples=10
        , grid=major
        ]
    \addplot {x/10};
    \addplot {x/1.0};
    \addplot {x/0.1};
    \legend{$R=10 \Omega$, $R=1 \Omega$, $R=0.1 \Omega$};
    \end{axis}

\end{tikzpicture}
\end{center}
\end{figure}

\section{A resistor in series with a battery of -80
mV}\label{a-resistor-in-series-with-a-battery-of--80-mv}

\begin{circuitikz}[american voltages]
\centering
\draw (0, 0) to[R, l=$R$] (3,0) to[V, l=$-80 mV$] (6,0);
\end{circuitikz}

\begin{tikzpicture}[scale=1]
    \begin{axis}[
        title={$V_R(t) = R * i_R(t) + 0.080 $}
        , xlabel=$V_R(t)$
        , ylabel=$i_R(t)$
        , legend pos=north west
        , domain=-0.1:0.1
        , samples=10
        , grid=major
        ]
    \addplot {(x+0.080)/10};
    \addplot {(x+0.080)/0.5};
    \legend{$R=10 \Omega$, $R=0.5 \Omega$};
    \end{axis}
\end{tikzpicture}

\section{A resistor in series with a battery of -80 mV, where the
resistance is\\ infinity below -60 mV, is 2R from -60 to 0 mV, and is R
above 0
mV}\label{a-resistor-in-series-with-a-battery-of--80-mv-where-the-resistance-is-infinity-below--60-mv-is-2r-from--60-to-0-mv-and-is-r-above-0-mv}

\pgfmathdeclarefunction{Resistance}{2}{%
    \pgfmathparse{%
        (and(1, #2<-0.060)*10e50+and(#2>=-0.060,#2<0)*2*#1+and(1, #2>0)*#1) }
}

\begin{tikzpicture}[scale=1]
    \begin{axis}[
        title={$V_R(t) = R(V_R) * i_R(t) - 0.080 $}
        , xlabel=$V_R(t)$
        , ylabel=$i_R(t)$
        , legend pos=north west
        , domain=-0.1:0.1
        , samples=20
        , grid=major
        ]
    \addplot {(x+0.080)/Resistance(10,x)};
    \addplot {(x+0.080)/Resistance(0.5,x)};
    \legend{$R=10 \Omega$, $R=0.5 \Omega$};
    \end{axis}
\end{tikzpicture}

\section{A potassium channel}\label{a-potassium-channel}

Assume no inactivation. Assume that the K channel is closed below -60 mV
and has a reversal potential of -80 mV and has an open conductance of R,
for all potentials above 0 mV.

\pgfmathdeclarefunction{PotChan}{2}{%
    \pgfmathparse{%
        (and(1, #2<-0.060)*10e50+and(#2>=-0.060,#2<0)*20*#1+and(1, #2>0)*#1) }
}

No information is given what is the conductance of channel when membrane
potential is between -60 mV and 0 mV. Assuming it is more than R, the
curve is more or less likely to be the one just above.

\begin{tikzpicture}[scale=1]
    \begin{axis}[
        title={$V_R(t) = R(V_R) * i_R(t) - 0.080 $}
        , xlabel=$V_R(t)$
        , ylabel=$i_R(t)$
        , legend pos=north west
        , domain=-0.1:0.1
        , samples=20
        , grid=major
        ]
    \addplot {(x+0.080)/PotChan(10,x)};
    \addplot {(x+0.080)/PotChan(0.5,x)};
    \legend{$R=10 \Omega$, $R=0.5 \Omega$};
    \end{axis}
\end{tikzpicture}

\end{document}
