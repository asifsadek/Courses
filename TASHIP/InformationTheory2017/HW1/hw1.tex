% From here  https://raw.githubusercontent.com/jgm/pandoc-templates/master/default.beamer
\documentclass[]{article}
\usepackage{mathtools,amsmath,amssymb,amsthm,mathrsfs}
\usepackage{commath}
\usepackage{lmodern}
\usepackage[sc,osf]{mathpazo}
%\usepackage{circuitikz}
\makeatletter
\@ifpackageloaded{tex4ht}{
    \def\pgfsysdriver{pgfsys-tex4ht.def}
}
\makeatother
\usepackage{pgfplots}
\usepackage{pgfplotstable}
\usepackage[multidot]{grffile}

%% Include newcommands.
\IfFileExists{tex_newcommands.tex}{
    \input{tex_newcommands.tex}
}

\IfFileExists{dot2texi.sty}{
    \usepackage{dot2texi}
}{ }

\usepackage{pgf,tikz}
\usetikzlibrary{shapes,backgrounds,positioning,matrix,decorations}

\providecommand{\tightlist}{%
\setlength{\itemsep}{0pt}\setlength{\parskip}{0pt}}

\usepackage{siunitx}
\usepackage{ifxetex,ifluatex}
\usepackage{listings}
\lstset{ 
    basicstyle=\ttfamily,
    numbers=left,
    numberstyle=\footnotesize,
    stepnumber=2,
    keywordstyle=\color[rgb]{0.13,0.29,0.53}\bfseries,
    stringstyle=\color[rgb]{0.31,0.60,0.02},
    commentstyle=\color[rgb]{0.56,0.35,0.01}\itshape,
    numbersep=5pt,
    backgroundcolor=\color{black!10},
    tabsize=2,
    breaklines=true,
    linewidth=\textwidth
}
% \usepackage[xindyo,acronym,nomain,toc]{glossaries}
% \makeglossaries
%\usepackage[xindy]{imakeidx}
%\makeindex
\setlength{\parskip}{3mm}
\newtheorem{axiom}{Axiom}
\newtheorem{definition}{Definition}
%\newtheorem{comment}{Comment}
\newtheorem{example}{Example}
\newtheorem{lemma}{Lemma}
\newtheorem{corollary}{Corollary}
\newtheorem{property}{Property}
\newtheorem{problem}{Problem}
\newtheorem{remark}{Remark}
\newtheorem{theorem}{Theorem}
\newtheorem{script}{Script}
\usepackage{fixltx2e} % provides \textsubscript
% use upquote if available, for straight quotes in verbatim environments
\IfFileExists{upquote.sty}{\usepackage{upquote}}{}
\ifnum 0\ifxetex 1\fi\ifluatex 1\fi=0 % if pdftex
  \usepackage[utf8]{inputenc}
\else % if luatex or xelatex
  \ifxetex
    \usepackage{mathspec}
    \usepackage{xltxtra,xunicode}
  \else
    \usepackage{fontspec}
  \fi
  \defaultfontfeatures{Mapping=tex-text,Scale=MatchLowercase}
  \newcommand{\euro}{€}
\fi
% use microtype if available
\IfFileExists{microtype.sty}{\usepackage{microtype}}{}
\usepackage[right=5cm, marginparwidth=4cm]{geometry}
\usepackage{longtable,booktabs}
\ifxetex
  \usepackage[setpagesize=false, % page size defined by xetex
              unicode=false, % unicode breaks when used with xetex
              xetex]{hyperref}
\else
  \usepackage[unicode=true]{hyperref}
\fi
\hypersetup{breaklinks=true,
            bookmarks=true,
            pdfauthor={Dilawar Singh},
            pdftitle={Solution to Homework 1},
            colorlinks=true,
            citecolor=blue,
            urlcolor=blue,
            linkcolor=magenta,
            pdfborder={0 0 0}}
\urlstyle{same}  % don't use monospace font for urls
\setlength{\parindent}{0pt}
\setlength{\parskip}{6pt plus 2pt minus 1pt}
\setlength{\emergencystretch}{3em}  % prevent overfull lines
\setcounter{secnumdepth}{5}

\title{Solution to Homework 1}
\author{Dilawar Singh}
\date{\today}
\usepackage{pgfplots}
\usepackage{subfig}
\AtBeginDocument{%
\renewcommand*\figurename{Figure}
\renewcommand*\tablename{Table}
}
\AtBeginDocument{%
\renewcommand*\listfigurename{List of Figures}
\renewcommand*\listtablename{List of Tables}
}
\usepackage{float}
\floatstyle{ruled}
\makeatletter
\@ifundefined{c@chapter}{\newfloat{codelisting}{h}{lop}}{\newfloat{codelisting}{h}{lop}[chapter]}
\makeatother
\floatname{codelisting}{Listing}
\newcommand*\listoflistings{\listof{codelisting}{List of Listings}}

\begin{document}
\maketitle

\section{Coding}\label{coding}

Consider a horse race where the probabilities of winning are given by
negative powers of 2 (1/2, 1/32, etc.). Suppose the lowest possible
probability for any horse is \(2^{-m}\), and that there are \(n_i\)
horses in each probability class \(2^{-i}\).

\begin{enumerate}
\def\labelenumi{\alph{enumi}.}
\tightlist
\item
  What equation must the \(n_i\) satisfy?
\item
  Give 3 possible sets of \(\{n_i\}\) for an 8 horse race (other than
  the uniform case).
\item
  Using the entropy concept, calculate the expected codeword length for
  each set.
\item
  For each set, present an efficient binary coding scheme that achieves
  optimality.
\end{enumerate}

\textbf{Solution}

\textbf{a.} Total probability (probability of all events) must adds up
to 1 i.e. \(\sum n_i * 2^{-i} = 1\)

\textbf{b, c.} One case is trivial that all 8 horses belongs to
probability class \(2^{-3}\). Each horse has the probability of winning
\(\frac{1}{8}\).

Script \texttt{./problem1.hs} generates some solutions (probably all
possible) to this problem. They are listed below.

\begin{longtable}[]{@{}llllll@{}}
\toprule
\(1/2\) & \(1/4\) & \(1/8\) & \(1/16\) & \(1/32\) &
Entropy\tabularnewline
\midrule
\endhead
0 & 0 & 8 & 0 & 0 & 3.0\tabularnewline
0 & 1 & 5 & 2 & 0 & 2.875\tabularnewline
0 & 2 & 2 & 4 & 0 & 2.75\tabularnewline
0 & 2 & 3 & 1 & 2 & 2.6875\tabularnewline
0 & 3 & 0 & 3 & 2 & 2.5625\tabularnewline
0 & 3 & 1 & 0 & 4 & 2.5\tabularnewline
1 & 0 & 1 & 6 & 0 & 2.375\tabularnewline
1 & 0 & 2 & 3 & 2 & 2.3125\tabularnewline
1 & 0 & 3 & 0 & 4 & 2.25\tabularnewline
1 & 1 & 0 & 2 & 4 & 2.125\tabularnewline
\bottomrule
\end{longtable}

Code are not given. Your code must be decodable and its entropy should
be near or equal to calculated entropy given in table above.

\section{{[}CT 2.13{]} Coin weighing.}\label{ct-2.13-coin-weighing.}

Suppose one has n coins, among which there may or may not be one
counterfeit coin. If there is a counterfiet coin, it may be either
heavier or lighter than the other coins. The coins are to be weighed by
a balance.

\begin{enumerate}
\def\labelenumi{\alph{enumi}.}
\tightlist
\item
  Find an upper bound on the number of coins n so that k weighings will
  find the counterfiet coin (if any) and correctly declare it to be
  heavier or lighter.
\item
  Suppose you have k = 3 weighings and 12 coins. How many coins must you
  weigh against each other in the first round?
\item
  (Optional, and hard) Find the full strategy for the 12 coin case.
\end{enumerate}

\textbf{Solution}

Here is smaller problem to give you some insight. Lets say two coins,
\(a\) and \(a'\) where \(a'\) is counterfiet. Assume that you also have
an additional coin \(b\) which is not counterfiet. How many weighings to
find out which one is counterfiet? One. You compare \(b\) with \(a\),
both weight the same, you declare that \(a'\) is counterfiet. You can't
tell if \(a'\) is lighter or heavier. Howver if you weight \(b\) with
\(a'\), you can tell that \(a'\) is counterfiet and if lighter or
heavier. The idea is to use the information from previous weighings to
reduce the state space.

My strategy is following. Divide the coins into 3 parts a, b, and c and
weigh any two of them. Lets say that we weigh a and b and they are
equal, then counterfiet coin in in c. Otherwise, counterfiet coin in in
either a or b. So if \(a\) and \(b\) weigh the same then counterfiet
coin is in the rest of one-third of coins; otherwise they are in \(a\)
and \(b\) (two-third of coins).

Lets draw \(k\) coins. The probability that counterfiet coins is in
these k coins is \(p(k)\). It is easy to show that
\(p(k) = \frac{k}{12}\). With probability \(p(k)\) (that counterfiet
coin is in selected k coins), we reduce the state space by one-third;
and with probability \(1-p(k)\) one-third. One average we reduce the
state space by \(p(k)(N-2k)+(1-p(k))(2k)\). What value of \(k\) maximize
this function?

\begin{tikzpicture}[scale=1]
    \begin{axis}[ xlabel=k, ylabel=State Space reduction, domain=0:9
            , title = {It is maximum at k=4}
        ]
        \addplot [color=blue,] plot { x/12 * (12-2*x) + (1 - x/12) * 2*x };
    \end{axis}
\end{tikzpicture}

\end{document}
