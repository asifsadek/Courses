\documentclass[solutions]{exam}
\title{Homework 3}
\author{Dilawar Singh}
\date{\today}

% Set up counters for problems and subsections
\begin{document}
\maketitle

\begin{questions}

\question Decoding

Consider the following set of codewords: $(A,B,C,D,E,F,G,H) = (01, 11, 001,
0000, 0001, 1001, 1010, 1011)$.

\begin{parts}
\part Is this an instantaneous (prefix) code?
\part Verify that it satisfies the Kraft inequality
\part Construct a string which has no meaning under this system
\end{parts}

\question Strange but typical sequences. We will show that a sequence can be typical by the
“probability label” used in class definition, but still have a very skewed proportion of
each letter!

Imagine strings on the alphabet $\{A,B,C\}$, with underlying probabilities
$p=\{1/2,1/3,1/6\}$. You see a string $S$ of length $N$ with observed letter
frequencies (divided by $N$) given by $q = {a, b, c}$, with $a + b + c = 1$. T
he most probable sequence is \texttt{AAAAAAA.....}.
Typical sequences include those around the point $\{1/2,1/3,1/6\}$.

\begin{parts}
    \part What is H(p)?
    \part What are the values of a and b such that a + b = 1, where the value of
    $-\frac{1}{n} \log (Pr(S))$ is very close to $H$? I.e. what sequences with
    no ‘$C$’s at all have probabilities similar to that of the typical sequence?
\end{parts}

\question Ternary plots
You can plot any three non-negative numbers that add up to 1 in a “ternary plot” with
x- and y-coordinates given by $(b + c/2, 3c/2)$. As $a, b, c$ range over their
allowed values, this will make an equilateral triangle,
where $x, y = (0,0)$ is equivalent to ${a = 1, b = 0, c = 0}$ (bottom left corner)
$x, y = (1,0)$ is equivalent to ${a = 0, b = 1, c = 0}$ (bottom right corner)
$x, y =(1/2, \sqrt{3}/2)$, is equivalent to ${a = 0, b = 0, c = 1}$ (top
corner).

On a ternary plot, make a contour plot of probabilities of sequences assuming
underlying probabilities $=(1/2,1/3,1/6)$. Mark the curve corresponding to the
set of sequences with probabilities similar to that of the typical sequence.
Hint: this can all be done by hand.

\end{questions}

\end{document}
