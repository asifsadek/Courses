\RequirePackage{luatex85,shellesc}
\documentclass[answers]{exam}
\usepackage{concrete}
\usepackage{libertine}
\usepackage{hyperref}
\usepackage{bashful}

\title{Homework 3}
\author{Dilawar Singh}
\date{\today}

% Set up counters for problems and subsections
\begin{document}
\maketitle

\begin{questions}

\question

\textbf{Decoding} Consider the following set of codewords: $(A,B,C,D,E,F,G,H) = (01, 11, 001, 0000, 0001, 1001, 1010, 1011)$.

\begin{parts}

    \part[3] Is this an instantaneous (prefix) code?
    \begin{solution}
        Yes.
    \end{solution}

    \part[2] Verify that it satisfies the Kraft inequality
    \begin{solution}
        $\sum_i r^{l_i}$ where $r=2$ (two alphabets) is 0.9375.
    \end{solution}

    \part[3] Construct a string which has no meaning under this system

    \begin{solution} Here is one.
    \begin{verbatim}
    01110110011000110111
     A B A   F   x
    \end{verbatim}
    \end{solution}

\end{parts}

Script
\url{http://github.com/dilawar/courses/raw/master/TASHIP/InformationTheory2019/HW3/prob1.py}
solves these problems. Following is a sample run. This script requires Python
3.6 or higher version.

\begin{verbatim}
=== (a)
Code is instantaneous
=== (b)
Satifies Kraft inequality since 0.9375<1.
=== (c)
10010100011
   F A   E
Successfully decoded
011011010110010
 A   H A A   F
Successfully decoded
0000101101
   D   H A
Successfully decoded
110111001010100
 B A B  C A A
Successfully decoded
010111011101
 A A B A B A
Successfully decoded
00011100011101000
   E B   E B A
Successfully decoded
0101111110101001011
 A A B B   G   F A
Successfully decoded
1111111000100011
 B B B   x
Failed to decode
\end{verbatim}


\question 

\textbf{Strange but typical sequences} We will show that a sequence can be
typical by the “probability label” used in class definition, but still have a
very skewed proportion of each letter!

Imagine strings on the alphabet $\{A,B,C\}$, with underlying probabilities
$p=\{1/2,1/3,1/6\}$. You see a string $S$ of length $N$ with observed letter
frequencies (divided by $N$) given by $q = {a, b, c}$, with $a + b + c = 1$. T
he most probable sequence is \texttt{AAAAAAA.....}.
Typical sequences include those around the point $\{1/2,1/3,1/6\}$.

\begin{parts}
    \part What is H(p)?
    \part What are the values of a and b such that a + b = 1, where the value of
    $-\frac{1}{n} \log (Pr(S))$ is very close to $H$? I.e. what sequences with
    no ‘$C$’s at all have probabilities similar to that of the typical sequence?
\end{parts}

\bonusquestion[10] \textbf{Ternary plots}
You can plot any three non-negative numbers that add up to 1 in a “ternary plot” with
x- and y-coordinates given by $(b + c/2, 3c/2)$. As $a, b, c$ range over their
allowed values, this will make an equilateral triangle,
where $x, y = (0,0)$ is equivalent to ${a = 1, b = 0, c = 0}$ (bottom left corner)
$x, y = (1,0)$ is equivalent to ${a = 0, b = 1, c = 0}$ (bottom right corner)
$x, y =(1/2, \sqrt{3}/2)$, is equivalent to ${a = 0, b = 0, c = 1}$ (top
corner).

On a ternary plot, make a contour plot of probabilities of sequences assuming
underlying probabilities $=(1/2,1/3,1/6)$. Mark the curve corresponding to the
set of sequences with probabilities similar to that of the typical sequence.
Hint: this can all be done by hand.

\end{questions}

\end{document}
