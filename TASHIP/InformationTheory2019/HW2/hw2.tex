\RequirePackage{luatex85,shellesc}
\documentclass[answers]{exam} 
\usepackage{amssymb,amsmath}
\usepackage{hyperref}
\usepackage{longtable,booktabs}
\usepackage{bashful}

\usepackage{pgfplotstable}
\title{Homework 2}
\author{Dilawar Singh}
\date{\today}

\begin{document}
\maketitle

\begin{questions}

\question[9] Joint probability distribution.

\begin{solution}

\pgfplotstabletypeset[col sep=tab
   , every head row/.style={before row=\toprule, after row=\midrule}
   , columns/Entropy/.style={string type,column type=r}
    ]{sol1.tsv}

These are generated by 
\href{http://github.com/dilawar/courses/raw/master/TASHIP/InformationTheory2019/HW2/prob1.py}{this
python script}. I don't trust my hand calculation!

\end{solution}

\question In class we had defined a stringently typical sequence as one
containing exactly as many occurrences of each symbol as expected. Let’s find
out \textbf{(a)} what the probability of each such sequence is and \textbf{(b)}
how many such sequences there are \textit{exactly} (i.e. not using Stirling’s
approximation). Then let’s see how much of the total probability space is
occupied by these typical sequences.  Consider a DNA sequence of length 8
generated iid from the distribution

\[ \Pr(A,T,G,C) = (1/2,1/4,1/8,1/8) \].

\begin{parts}
\part[3] What is the single most probable sequence? What is its probability of occurrence?
\begin{solution}
    Single most probable sequence is the one in which every element is most
    probable i.e., all elements are \texttt{A}. Therefore \texttt{AAAAAAAA} is
    the most probable sequence. Its probability is
    $\left(\frac{1}{2}\right)^8=\pgfmathsetmacro\res{2^-8}\res$.
\end{solution}
\part[3] What is the probability of a given ‘stringently typical’ sequence, defined as one in
which letters occur precisely as often as expected?
\begin{solution}
    In a sequence of length $n$, a stringently typical sequence will have
    alphabet \texttt{x} occurring exactly $\Pr(x)*n$ times; i.e.,
    \texttt{A} occurring exactly $n/2$ times, \texttt{T} occurring exactly $n/4$
    times, \texttt{C} and \texttt{G} occurring exactly $n/8$ times. The
    probability of such a sequence is \( \prod_i \Pr(x_i)^{\Pr(x_i)n} \).

    In the case of sequence length 8, the answer is $(1/2)^4 (1/4)^2 (1/8)^1
    (1/8)^1=\pgfmathsetmacro\pString{(2^-4)*(4^-2)*(8^-1)*(8^-1)}\pString$.
\end{solution}

\part[3] How many stringently typical sequences are there (exact answer required)?
\begin{solution}
    How many ways you can choose alphabet $x_i$, $n\Pr(x_i)$ times? Number of
    sequences which has 4 \texttt{A}, 2 \texttt{T} and 1 \texttt{C} and
    \texttt{G} each i.e., ${8\choose 4}{4\choose2}{2\choose 1}{1\choose 1}=280$
\end{solution}

\part[1] What is the total probability of getting some stringently typical sequence?
\begin{solution}
    \pgfmathsetmacro\res{280/16/16/64}\res. 
\end{solution}
\part[5] Redo the whole calculation if the length is 16. What is the total probability of
getting a stringently typical sequence? Are we converging to 1?
\begin{solution}
    Number of total stringently typical sequences are \( {16\choose 8}{8\choose
    4}{4\choose 2}{2\choose 2}=5405400\) and the probability of 1 such sequence
    is $2^{-8}4^{-4}8^{-2}8^{-2}=4\times 10^{-9}$ and the product 0.002013. It
    is less the case of sequence length of 8. Definitely we are not converging
    to 1.
\end{solution}
\end{parts}

\bonusquestion[10] 
Revisit Problem 2 from HW1, but now armed with the definition of mutual
information. Set up a table where each row Y corresponds to the
potential outcome of the experiment (L, B, R for left-heavy, balanced,
or right-heavy) and each column X corresponds to each of the 25 equally
likely options (1 -- 12 for one coin heavy; 13 -- 24 for one coin light;
25 for no counterfeit).

Suppose the first weighing involves two sets of a coins, for a = 1,
\ldots{} , 6 . Set up the correct joint distribution in each case.
Calculate the mutual information I(X;Y). This tells you how many bits of
information your measurement provides. What is the most informative
measurement?

\begin{solution}

Lets enumerate all possible states. A coin can be of 3 types: \texttt{.},
\texttt{l}, \texttt{h}. If first coin was heavy then we have encode the state as
\texttt{h...........}, if the 5th coin is light then we have
\texttt{....l.......} and so on. So there are total 36 states. But when a normal
coin \texttt{.} is chosen for any places, then the state is always
\texttt{............}. There are 12 such states and indistinguishable. These
states can be lumped into one state \texttt{NNNNNNNNNNNN}. Therefore we have 36
- 11 = 25 distinguishable macro states.

Lets weight first 2 coins with third and fourth coins.  There are three
possibilities: left heavy (\texttt{LH}), right heavy ( \texttt{RH}), or both
equal (\texttt{EQ}). What is the probability of these outcome? The probability
of \texttt{EQ} is equal to the probability that all 4 coins are normal coins.
And it is equal to $\frac{11}{12}\frac{10}{11}\frac{10}{11}\frac{9}{10}$ when
there is a fake coin and 1 when there is none. Assuming that half the time there
is fake coin, we take the weighted average.

My solution is in this python script
\url{https://github.com/dilawar/courses/raw/master/TASHIP/InformationTheory2019/HW2/prob3.py}.
But I don't think that it is correct. Let's see what you guys have done.

\bash[stdout]
python prob3.py
\END

\end{solution}


\end{questions}

\end{document}
