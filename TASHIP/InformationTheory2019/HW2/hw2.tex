\documentclass[answers]{exam} 
\usepackage{amssymb,amsmath}
\usepackage{hyperref}
\usepackage{longtable,booktabs}
\title{Homework 2}
\author{Dilawar Singh}
\date{\today}

\begin{document}
\maketitle

\begin{questions}

\question[9] Joint probability distribution.

\begin{solution}

\begin{longtable}[]{@{}ll@{}}
\toprule
\endhead
H(X,Y) & 1.5\tabularnewline
H(X) & 0.8112781\tabularnewline
H(Y) & 1.0\tabularnewline
H(X\textbar{}Y) & 0.5\tabularnewline
H(Y\textbar{}X) & 0.68872184\tabularnewline
H(X) + H(Y) - H(X,Y) & 0.3112781\tabularnewline
H(X) - H(X\textbar{}Y) & 0.3112781\tabularnewline
H(Y) - H(Y\textbar{}X) & 0.31127816\tabularnewline
I(X;Y) H(X) - H(X,Y) & 0.3112781\tabularnewline
\bottomrule
\end{longtable}

These are generated by 
\href{http://github.com/dilawar/courses/raw/master/TASHIP/InformationTheory2019/HW2/prob1.py}{this
python script}
\end{solution}

\question
In class we had defined a stringently typical sequence as one containing exactly as
many occurrences of each symbol as expected. Let’s find out (a) what the probability of
each such sequence is and (b) how many such sequences there are \textit{exactly} (i.e. not using
Stirling’s approximation). Then let’s see how much of the total probability space is
occupied by these typical sequences.
Consider a DNA sequence of length 8 generated iid from the distribution

\[ \Pr(A,T,G,C) = (1/2,1/4,1/8,1/8) \].

\begin{parts}
\part[2] What is the single most probable sequence? What is its probability of occurrence?
\part[2] What is the probability of a given ‘stringently typical’ sequence, defined as one in
which letters occur precisely as often as expected?
\part[3] How many stringently typical sequences are there (exact answer required)?
\part[3] What is the total probability of getting some stringently typical sequence?
\part[10] Redo the whole calculation if the length is 16. What is the total probability of
getting a stringently typical sequence? Are we converging to 1?
\end{parts}

You should find that, as the sequences get longer, fewer and fewer of them are ‘typical’
by this definition. This motivates the new definition of typical sequence we will make
this week.

\bonusquestion[10] 
Revisit Problem 2 from HW1, but now armed with the definition of mutual
information. Set up a table where each row Y corresponds to the
potential outcome of the experiment (L, B, R for left-heavy, balanced,
or right-heavy) and each column X corresponds to each of the 25 equally
likely options (1 -- 12 for one coin heavy; 13 -- 24 for one coin light;
25 for no counterfeit).

Suppose the first weighing involves two sets of a coins, for a = 1,
\ldots{} , 6 . Set up the correct joint distribution in each case.
Calculate the mutual information I(X;Y). This tells you how many bits of
information your measurement provides. What is the most informative
measurement?

\begin{solution}
Lets do this problem hard way. We are going to index them. First coins
will be \(C_0\), second \(C_1\), and so on. And the last 12th coin is
\(C_{11}\). Next we arrange all coin in a line
\(C_0 C_1 \ldots C_{11}\). A coin \(C_i\) could be one to the three
types: light \texttt{L}, heavy \texttt{H}, or normal \texttt{N}.

Lets enumerate all possible states. There are 3 ways (\texttt{N},
\texttt{L}, \texttt{H}) in which we can choose any coin \(C_i\) e.g.~if
first coin is chosen \texttt{H} (rest normal) then we have
\texttt{HNNNNNNNNNNN}, if the 5th coins is chosen \texttt{L} then we
have \texttt{NNNNLNNNNNNN}, and if 8th coins is chosen \texttt{N}
i.e.~\texttt{NNNNNNNNNNNN}. So there are total 36 states. But when a
normal coin \texttt{N} is choosen for any \(C_i\), then the state is
always \texttt{NNNNNNNNNNNN} all coins are normal. There are 12 such
states and indistinguishable. These states can be lumped into one state
\texttt{NNNNNNNNNNNN}. Therefore we have 36 - 11 = 25 distinguishable
macro states.

Lets weight first 2 coins \(C_1C_2\) with third and fourth coins
\(C_3C_4\). There are three possibilities that \(C_1C_2 > C_3C_4\),
\(C_1C_2 < C_3C_4\), or \(C_1C_2 = C_3C_4\). Now trick is to enumerate
possible states for each of these outcomes.
\end{solution}

\end{questions}

\end{document}
