\documentclass[twocolumn]{article}
\usepackage[left=35mm,margin=20mm]{geometry}
\usepackage{ifthen,empheq}
\date{\today}
\usepackage{fancyhdr}
\usepackage{multicol}
\usepackage{enumitem}
\usepackage{pgfplots}
\setlength{\headheight}{15pt}

\lhead{NAADP mobilizes calcium from acidic organelles through two-pore channels}
\rhead{Dilawar Singh}

\lfoot{}\cfoot{\thepage}\rfoot{}
\pagestyle{fancy}

\ifx\pdfoutput\undefined                         %LaTeX
  \RequirePackage[ps2pdf,bookmarks=true]{hyperref}
  \hypersetup{ %
    pdfauthor   = {\@author},
    pdftitle    = {\@title},
    pdfcreator  = {LaTeX with hyperref package},
    pdfproducer = {dvips + ps2pdf}
  }
\else                                            %PDFLaTeX
  \RequirePackage[pdftex,bookmarks=true]{hyperref}
  \hypersetup{ %
    pdfauthor   = {\@author},
    pdftitle    = {\@title},
    pdfcreator  = {LaTeX with hyperref package},
    pdfproducer = {dvips + ps2pdf}
  }
\pdfadjustspacing=1
\fi

% Set up counters for problems and subsections

\newcounter{ProblemNum}
\newcounter{SubProblemNum}[ProblemNum]

\renewcommand{\theProblemNum}{\arabic{ProblemNum}}
\renewcommand{\theSubProblemNum}{\alph{SubProblemNum}}

\newcommand*{\anyproblem}[1]{\newpage\subsection*{#1}}
\newcommand*{\problem}[1]{\stepcounter{ProblemNum} %
   \anyproblem{Problem \theProblemNum. \; #1}}
\newcommand*{\soln}[1]{\subsubsection*{#1}}
\newcommand*{\solution}{\soln{Solution}}
\renewcommand*{\part}{\stepcounter{SubProblemNum} %
  \soln{Part (\theSubProblemNum)}}

\renewcommand{\theenumi}{(\alph{enumi})}
\renewcommand{\labelenumi}{\theenumi}
\renewcommand{\theenumii}{\roman{enumii}}

% Glossaries
\usepackage[xindy]{glossaries}
\usepackage{tikz}
\usetikzlibrary{graphs,graphdrawing}
\usetikzlibrary{quotes,arrows,shapes,arrows.meta}
\usegdlibrary{layered,force}

\begin{document}
\makeglossaries

\paragraph{Summary of $Ca^{++}$ release from various stores}  This is shown in figure \ref{fig:ca_release}.

\begin{figure}[h]
    \centering
    %% NOTE: This is to compl
    \begin{tikzpicture}[scale=1
        , every node/.style={}
        ]

        \graph [ nodes = {}, edges = {->,thick}
        , node distance = 1.5cm
        , sibling distance = 1.5cm
        , level distance = 1.5cm
        , layered layout 
        ]
        {
            "? stores" [rectangle,draw];
            "mobilization" [blue];
            "? stores" -> mobilization -> "higher $Ca^{++}$";

            "S/ER stores"[draw,rectangle] // { "$InsP_3R$", RyRs, OR[circle,draw] };

            % The process of mobilization is regulated by inositol-1,4,5
            % triPhosphate InsP_3, cyclic ADP ribose and NAADP.
            { NAADP, "$InsP_3$", "cADP ribose" } -> regulation[blue] 
                ->[-Turned Square]  mobilization;

            release [blue] -> "higher $Ca^{++}$";

            { "cADP ribose", "$InsP_3$" } ->[->] activation [blue] -> { "$InsP_3R$",
                RyRs } -> "OR"[circle,draw] -> release;


            NAADP ->[dotted, thin] "?"[blue] ->[dotted,thin] RyRs;

            "cross-talk"[rectangle,fill=blue,fill opacity=0.1] // { "cADP ribose", "$InsP_3$", NAADP };

        };
    \end{tikzpicture}    
    \caption{ The three main
molecules $InsP_3$, cyclic ADP ribose, and NAADP causes release/mobilization of
$Ca^{++}$ from various stores. The mechanism behind release from
sarcoplasmic/endoplasmic reticulum by $InsP_3$ and cADP ribose is well
established. This paper describes the NAADP targets. }
    \label{fig:ca_release}
\end{figure}

\paragraph{Does TCP2 forms a binding site for NAADP?}

To check if TCP2 is a potential binding site to NAADP,  TCP2 cell was incubated
with 0.2 nM $[{}^{32}P]$NAADP (in the presence and absence of unlabelled 100 uM
NAADP \footnote{In presence, the labelled NAADP is 2 ppm unlabelled NAADP}), the
specific binding of labelled NAADP to TCP2 cell is 3 times more than wildtype
cell membrane, which reverts back to wildtype levels after TCP2 was depleted.
{\bf TCP2 proteins provides binding sites to NAADP}. Ligand competition assays
shows two affinities to NAADP, stronger one being 1000 times higher. The author
could not rule out the possibility that interaction with
\href{http://beginw.ncbi.nlm.nih.gov/pubmed/16402902}{accessory protein} may be
necessary.

\paragraph{Does TPC2 mediate calcium release?} Caged NAADP was released by
light-flash and concentration of calcium were measured. A biphasic pulse
\begin{tikzpicture}
    \begin{axis}[hide axis, scale only axis, height=3ex]
    \addplot[mark=none] coordinates { (0, 0) (1, .1) (2, .2) (3, .3) (10, 1)
        (11, 6) (12, 2) (13,.5) (15, .4) (18, .2) 
        };
    \end{axis}
\end{tikzpicture} with slow ramp like phase, followed by quick surge and quick
decay is observed. The faster phase (both rise and fall) goes away when a
competitive inhibitor of $InsP_3R$s is added (i.e. the faster phase is due to
release from S/ER) meaning that initial slow ramp like release was due to NAADP.

\paragraph{Does NAADP also release calcium from S/ER?} Before adding the
competitive inhibitor of $InsP_3R$s, if one also complete deplete S/ER stores
(pretreatment by thapsigargin), one still sees a ramp like release of calcium by
NAADP. Therefore calcium release/mobilised by NAADP does not come from S/ER.

\paragraph{Where does calcium mobilised from?} By the logic of elimination, from
acidic organelles. Since NAADP bind to TPC2 strongly, it came from lysosomes.

\paragraph{Quantification of NAADP response} Further experiments were done to
quantify the calcium release/mobilization by NAADP. Low concentration (in range
of 0.1 nM, 1.5 molecules in 25 $um^3$ volume) or very high concentration  (1 mM)
did not cause any mobilization, however concentration in range of 1 nM (15
molecules in 25 $um^3$ volume) to 1 uM did  elicit the normal mobilization.
\footnote{The role of TCP1 and TCP3 is not ruled out. They probably release
    calcium from different organelles or probably a little from the
    endolysosomes as well.}

\paragraph{Inside a living system}
A TPC2 knocked-out mice was utilised to confirm the findings.
$Ca^{++}$-activated plasma membrane current was monitored. In wild-type mouse, a
cation current was observed. Its integration would confirm the dynamics
observed. In knocked-out mouse, no such current was observed.

\paragraph{What does is mean for cell (researchers)?}
For authors, interesting point is that the slow release by NAADP can be coupled
by many other known release from other calcium sources. This would enable cell
to construct more versatile "control" over intera-cellular  calcium
concentrations.  

For me, the interesting fact is that there are more than one pools of calcium
inside the cell. And NAADP effects type of them. NAADP concentrations correlated
with extracellular cellular stimulus; and calcium released by NAADP is a good
indicator of volume of lysosomes. If I were to study "how cell control their
sizes?", I'd definitely include this pathway to my list.

%\begin{figure}[h]
    %\centering
%\begin{tikzpicture}[scale=1
    %, every node/.style={}
    %]

    %\graph[ layered layout
        %, nodes = { }
        %, edges = {->,thick}
        %%, sibling distance = 1.5cm
        %%, node distance = 1.5cm
        %%, layer distance = 1.5cm
        %]
    %{
        %{"TCP2 membrane", "$[^{32}P]$NAADP"} -> "+"[draw,circle];
        
    %};
    
%\end{tikzpicture}    
%\caption{}
%\end{figure}


%% Gloassaries
\newglossaryentry{tcp} 
{
    name=TCP,
    description={voltage gated ion channel}
}
\newglossaryentry{TwoPoreChannel} 
{
    name=Two-pore channel,
    description={TODO}
}

\newglossaryentry{mobilization} 
{
    name=mobilization,
    description={Calcium release from intracellular stores into cell is called
        mobilization. This term is probably used to distinguish the process of
        calcium release into the cell.}
}

\paragraph{Notes}

\begin{itemize}[noitemsep,nolistsep]
    \item Three
        \href{http://biology.stackexchange.com/questions/10632/whats-a-non-allelic-gene}{non-allelic}
        TCP genes are present in most vertebrate species with TCP3 absent in
        most vertebrate. TCP3 is co-localised with LAMP2 (a gene). TCP1 and TCP2
        (mammalian TCP) co-localised poorly with LAMP2, and express strongly in
        endolysosomes.
\end{itemize}


\end{document}
