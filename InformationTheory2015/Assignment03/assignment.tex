\documentclass{article}
\usepackage[margin=20mm]{geometry}
\usepackage[english]{babel}
\usepackage[utf8]{inputenc}
\usepackage{fancyhdr}
\newlength\tindent
\setlength{\tindent}{\parindent}
\setlength{\parindent}{0pt}
\renewcommand{\indent}{\hspace*{\tindent}}
\pagestyle{fancy}
\fancyhf{}
\rhead{Assignment 3}
\chead{Dilawar Singh}
\lhead{Information Theory}
\rfoot{Page \thepage}
 
\begin{document}
 
Being too stringent with the definition of typical sequences 


\paragraph{Problem 1} \footnote{This problem explains why we had to go through all those
epsilons to get a good idea of what a typical sequence means.}

Consider a DNA sequence of length 8 generated iid from the distribution 
    
$$ \wp(A, T, G, C) = (\frac{1}{2}, \frac{1}{4}, \frac{1}{8}, \frac{1}{8})$$


1. What is the single most probable sequence? What is its probability of
occurrence?

\paragraph{Solution} The most probable sequence is $AAAAAAAA$ with probability
of occurrence $\frac{1}{2^8}$.

2. How many stringently typical sequences are there (exact answer required)?  
\paragraph{Solution} A (stringent) typical sequence $T_n$ is defined as
$-\frac{1}{n} log_2 p(T_n) - H(X) = 0$ which implies that $p(T_n) = 2^{-nH(x)}$.
Once can calculate that $H(X) = 1.75$ and $p(T_n) = 2^{-14}$. Since the
probability distribution has elements which are negative power of 2, the total
typical sequences are number of solutions to the equation $a + 2b + 3c + 3d =
14$ where $a,b,c,d$ are positive integers. We can put further constrains 
$0 \ge a \le 14, 0 \ge b \le 7, 0 \ge c \le 4, 0 \ge d \le 4$. The snippet in
appedix solves this problem. There are total 196 such sequences.

3. What is the total probability of getting some stringently typical sequence?
\paragraph{Solution} The probability of getting stringently typical sequence is
$\frac{196}{4^8} = 0.00299$.

4. Redo the whole calculation if the length is 16. What is the total probability
of getting a stringently typical sequence? Are we converging to 1?  


\paragraph{Entropy rate of a Markov process}

The Morse code uses dots, dashes, and two types of spaces: letter and word
spaces.  Spaces only occur as delimiters; that is, no space can follow another
space. What is the entropy rate achievable under these constraints?  Hint: take
a look at Shannon’s 1948 paper for a start, and Example 5.1.3 of C\&T, but use
different notation. Write it out as a Markov process whose states are [.], [-],
[L-space], [W-space]. Assume that all allowed outputs from any state are equally
likely.  

\paragraph{[Optional][C\&T 3.11] Read the C\&T section 4.3, Entropy Rate of a
    Random Walk}  What is the entropy rate of rooks, kings, and bishops on a 3x3
chessboard?  Remember there are 2 types of bishops.

\end{document}


